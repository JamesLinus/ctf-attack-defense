\newpage
\ESKDthisStyle{empty}
\paragraph{\hfill РЕФЕРАТ \hfill}
Курсовая работа содержит \ESKDtotal{page} страниц, \ESKDtotal{figure} рисунка, \ESKDtotal{table} таблицы, \ESKDtotal{bibitem} источников, \ESKDtotal{appendix} приложение.

%допилить ключевые слова
CTF, ATTACK-DEFENSE, СОРЕВНОВАНИЯ, KEVA, ЗАЩИТА ИНФОРМАЦИИ, ФЛАГИ, ПРИЕМКА ФЛАГОВ, SCOREBOARD, GIT, PYTHON, MONGODB, MONGOOSE, ЧЕКЕРЫ, HTML, CSS, JAVASCRIPT, КОМАНДА, API.

Цель работы --- создание программного комплекса, предназначенного для проведения соревнований в области информационной безопасности CTF.

Задачей, поставленной на курсовую работу, стало написание программного комплекса, имеющего следующие возможности: 
\begin{itemize}
\item инициализация с помощью графического интерфейса;
\item интеграция с существующим API и администраторской панелью; 
\item инициализация и работа проверяющей системы;
\item инициализация и работа приемки флагов;
\item инициализация и работа таблицы рейтинга.
\end{itemize}

Результаты работы в данной курсовой работе:

\begin{itemize}
\item выбраны язык программирования, база данных с учетом потребностей;
\item доработан существующий API и администраторская панель для управления соревнованиями; 
\item разработана архитектура ядра и организована работа с модулями;
\item разработан модуль работы с чекерами;
\item разработан модуль приемки флагов;
\item разработан модуль таблицы результатов;
\item произведено тестирование платформы в форме соревнований.
\end{itemize}

Пояснительная записка выполнена при помощи системы компьютерной вёрстки \LaTeX.
