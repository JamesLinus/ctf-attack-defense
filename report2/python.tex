Python --- высокоуровневый язык программирования общего назначения, ориентированный на повышение производительности разработчика и читаемости кода. Синтаксис ядра Python минималистичен. В то же время стандартная библиотека включает большой объём полезных функций \cite{python_wiki}. 


Основные архитектурные черты — динамическая типизация, автоматическое управление памятью, полная интроспекция, механизм обработки исключений, поддержка многопоточных вычислений и удобные высокоуровневые структуры данных. Код в Python организовывается в функции и классы, которые могут объединяться в модули.

Основными преимуществами языка программирования Python являются большое количество библиотек, кроссплатформенность, широкие возможности профилирования кода. 

Язык обладает чётким и последовательным синтаксисом, продуманной модульностью и масштабируемостью, благодаря чему исходный код написанных на Python программ легко читаем. При передаче аргументов в функции Python использует вызов по соиспользованию.

Разработка языка Python была начата в конце 1980-х годов сотрудником голландского института CWI Гвидо ван Россумом.