Для реализации программного комплекса для проведения соревнований в области информационной безопасноти была использована база данных MongoDB.

MongoDB --- документо-ориентированная система управления базами данных (СУБД) с открытым исходным кодом, не требующая описания схемы таблиц. Написана на языке C++. \cite{progit}.

Основные возможности MongoDB:
\begin{itemize}
\item документо-ориентированное хранение (JSON-подобная схема данных);
\item достаточно гибкий язык для формирования запросов;
\item динамические запросы;
\item поддержка индексов;
\item профилирование запросов;
\item быстрые обновления «на месте»;
\item эффективное хранение двоичных данных больших объёмов, например, фото и видео;
\item журналирование операций, модифицирующих данные в базе данных
\item поддержка отказоустойчивости и масштабируемости: асинхронная репликация, набор реплик и распределения базы данных на узлы;
\item может работать в соответствии с парадигмой MapReduce;
\item полнотекстовый поиск, в том числе на русском языке, с поддержкой морфологии.
\end{itemize}

Архитектура:

СУБД управляет наборами JSON-подобных документов, хранимых в двоичном виде в формате BSON. Хранение и поиск файлов в MongoDB происходит благодаря вызовам протокола GridFS. Подобно другим документо-ориентированным СУБД (CouchDB и др.), MongoDB не является реляционной СУБД. В СУБД:

\begin{itemize}
\item нет такого понятия, как «транзакция». Атомарность гарантируется только на уровне целого документа, то есть частичного обновления документа произойти не может;
\item Отсутствует понятие «изоляции». Любые данные, которые считываются одним клиентом, могут параллельно изменяться другим клиентом.
\end{itemize}

В MongoDB реализована асинхронная репликация в конфигурации «ведущий — ведомый» (англ. master — slave), основанная на передаче журнала изменений с ведущего узла на ведомые. Поддерживается автоматическое восстановление в случае выхода из строя ведущего узла. Серверы с запущенным процессом mongod должны образовать кворум, чтобы произошло автоматическое определение нового ведущего узла. Таким образом, если не используется специальный процесс-арбитр (процесс mongod, только участвующий в установке кворума, но не хранящий никаких данных), количество запущенных реплик должно быть нечётным.
