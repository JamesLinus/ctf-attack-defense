Основное назначение платформы - общение с сервисами команд-участников посредством отправки и проверки специально сгенерированных флагов на сервере. Предназначение этого модуля - работа с интерфейсом сервисов (чекерами).

\subsubsection{Принцип работы}
Игра делится на раунды, каждый раунд, обычно, длится 1 минуту (настраивается при инициализации). Интерфейс для работы модуля с сервисом называется чекер. За этот период модуль опрашивает с помощью чекеров все сервисы команд-участников. Структура взаимодействия модуля с сервисами представлена на рисунке 5.4.

\begin{figure}[ht!]
\center{\includegraphics[width=1.0\linewidth]{images/module_round_structure.png}}
\caption{Иерархическая структура взаимодействия модуля с сервисами}
\end{figure}

Опрос происходит в три этапа:
\begin{enumerate} 
\item Проверка работоспособности сервиса команды-участника;
\item Отправление флага на сервис; 
\item Проверка того, что флаг сохранен.
\end{enumerate}

На каждом этапе модуль ожидает в ответ один из четырех чисел: 101, 102, 103, 104. Эти числа также называются статусом сервиса команды-участника. Алгоритм работы представлен на рисунке 5.5.

\begin{figure}[ht!]
\center{\includegraphics[width=0.5\linewidth]{images/module_round_schema.png}}
\caption{Схема создания потоков}
\end{figure}


Число 101 соответствует успешной работе сервиса команды-участника. Число 102 означает, что сервис команды работает, но на каком-то этапе отвечает некорректно. Число 103 означает, сервис команды отвечает, но не работает из-за какой-либо ошибки. Число 104 означает, что сервис команды не отвечает на запрос. 

Если на каком-то этапе чекер возвращает число отличное от 101, дальнейшее выполнение программы прекращается, а статус чекера записывается в базу данных.

На 1 этапе каждому чекеру посылается адрес сервера команды-участника. 
На 2 этапе каждому чекеру посылается адрес сервера, идентификатор флага и сам флаг. 
На 3 этапе каждому чекеру посылается адрес сервера, идентификатор флага и сам флаг.

Алгоритм работы представлен на рисунке 5.6.
\begin{figure}[ht!]
\center{\includegraphics[width=0.3\linewidth]{images/module_round_toservice.png}}
\caption{Блок-схема работы с каждым сервисом каждой команды-участника}
\end{figure}


Для увеличения производительности, каждый опрос сервисов команд-участников помещается в новый поток. Это позволяет работать в асинхронном режиме. Поэтому нестабильная работа сервиса команды-участника или ошибка в работе чекера не повлияет на опрос других сервисов. Также каждому потоку задается ограничение по времени работы. Это позволяет своевременно завершать процессы, тем самым уменьшается нагрузка на процессор и на потребление памяти.

