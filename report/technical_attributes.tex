\subsection {Требования к аппаратному обеспечению}

Необходимо не менее двух компьютеров, находящихся в локальной сети (компьютеры команд-участников), а также сервер, на котором запускается платформа.

Минимальные системные требования для сервера:

\begin{itemize}
\item процессор 1ГГц Pentium 4;
\item оперативная память 512 Мб;
\item место на жёстком диске -- 9 Гб.
\end{itemize}

Минимальные системные требования для компьютеров команд-участников определяются исходя из написанных сервисов.

\subsection {Требования к программному обеспечению}
Для корректной работы разрабатываемого программного комплекса сервер должен работать под управлением ОС Ubuntu Linux. 
Также должны быть установлены следующие пакеты: Python 3.X, MongoDB, PHP 5, MySQL, Веб-сервер Apache 2 или Nginx.


Требования к ПО команд-участников выставляются уже при непосредственном проведении соревнований.

\subsection {Требования к сервисам}
Для каждого сервиса необходимо иметь следующие компоненты:

\begin{itemize}
\item сервис - программа, имеющая уязвимости;
\item чекер - программа, реализующая интерфейс работы между сервисом и проверяющей системой.
\end{itemize}

Чекер должен содержать в себе 3 метода:

\begin{enumerate}
\item Check --- проверка доступности сервиса. В параметрах передается адрес сервиса команды-участника;
\item Put --- отправление флага на сервис команды-участника. В параметрах передается адрес сервиса, идентификатор флага и флаг;
\item Get --- получение флага по его идентификатору. В параметрах передается адрес сервиса, идентификатор флага и флаг.
\end{enumerate}

Вызов чекера происходит при помощью вызова исполняемого файла с необходимыми параметрами. Первым параметром является название метода. 
В ответ чекер должен возвращать одно из четырех состояний, обозначенного кодами 101-104:
\begin{itemize}
\item 101 --- сервис команды работает и работает корректно;
\item 102 --- сервис команды работает, но на каком-то этапе отвечает некорректно;
\item 103 --- сервис команды отвечает, но не работает из-за какой-либо ошибки; 
\item 104 --- сервис команды не отвечает на запрос.
\end{itemize}


\clearpage
Ниже приведен шаблон кода программы чекера. Шаблон написан на языке Python.



\lstinputlisting[language=Python]{individual_reports/programs/python_sample.py}
\clearpage