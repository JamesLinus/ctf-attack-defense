Для разработки программного комплекса для проведения компьютерной экспертизы было решено использовать Git.

Git --- распределённая система управления версиями файлов. Проект был создан Линусом Торвальдсом для управления разработкой ядра Linux  как противоположность  системе управления версиями Subversion (также известная как «SVN») \cite{progit}.

При работе над одним проектом команде разработчикоа необходим инструмент для совместного написания, бэкапирования и тестирования программного обеспечения. Используя Git, мы имеем:
\begin{itemize}
\item возможность удаленной работы с исходными кодами;
\item возможность создавать свои ветки, не мешая при этом другим разработчикам;
\item доступ к последним изменениям в коде, т.к. все исходники хранятся на сервере git.keva.su;
\item исходные коды защищены, доступ к ним можно получить лишь имея RSA-ключ;
\item возможность откатиться к любой стабильной стадии проекта.
\end{itemize}

Основные постулаты работы с кодом в системе Git:

\begin{itemize}
\item каждая задача решается в своей ветке;
\item необходимо делать коммит как только был получен осмысленный результат;
\item ветка master мержится не разработчиком, а вторым человеком, который производит вычитку и тестирование изменения;
\item все коммиты должны быть осмысленно подписаны/прокомментированы.
\end{itemize}

Для работы над проектом проектной группой был поднят собственный репозиторий на сервере git.keva.su.
Адреса репозиториев следующие:

Исходные файлы проекта:

git clone git@git.keva.su:gpo.git gpo.git

Репозиторий для тестирования проекта:

git clone git@git.keva.su:gpo-testdata.git gpo-testdata.git
