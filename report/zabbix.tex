
Zabbix ---  свободная система мониторинга и отслеживания статусов разнообразных сервисов компьютерной сети, серверов и сетевого оборудования. 

Архитектура Zabbix:
\begin{itemize}
\item Zabbix-сервер — это ядро программного обеспечения Zabbix. Сервер может удаленно проверять сетевые сервисы, является хранилищем, в котором находятся все конфигурационные, статистические и оперативные данные.
\item Zabbix-прокси — собирает данные о производительности и доступности от имени Zabbix сервера. Все собранные данные заносятся в буфер на локальном уровне и передаются Zabbix серверу, к которому принадлежит прокси-сервер.
\item Zabbix-агент — контроль локальных ресурсов и приложений (таких как жесткие диски, память, статистика процессора и т. д.) на сетевых системах. Эти системы должны работать с запущенным Zabbix агентом.
\item Веб-интерфейс — интерфейс является частью Zabbix сервера, и, как правило (но не обязательно), запущен на том же физическом сервере, что и Zabbix сервер. Работает на PHP, требует веб-сервер (например, Apache).
\end{itemize}