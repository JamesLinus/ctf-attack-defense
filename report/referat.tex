\newpage
\ESKDthisStyle{empty}
\paragraph{\hfill РЕФЕРАТ \hfill}

Курсовая работа содержит \ESKDtotal{page} страниц, \ESKDtotal{figure} рисунков, \ESKDtotal{bibitem} источников, \ESKDtotal{appendix} приложение.

%допилить ключевые слова
CTF, ATTACK-DEFENSE, СОРЕВНОВАНИЯ, KEVA, ЗАЩИТА ИНФОРМАЦИИ, ФЛАГИ, ПРИЕМКА ФЛАГОВ, SCOREBOARD, GIT, PYTHON, MONGODB, MONGOOSE, ZABBIX, OPTPARSE, ЧЕКЕРЫ, HTML, CSS, JAVASCRIPT, КОМАНДА, API.

Цель работы --- доработка программного комплекса, предназначенного для проведения соревнований в области информационной безопасности CTF.

Задачей, поставленной на данный семестр, стало написание программного комплекса, имеющего следующие возможности: 
\begin{itemize}
\item инициализация с помощью графического интерфейса;
\item интеграция с существующим API и администраторской панелью; 
\item инициализация и работа проверяющей системы;
\item инициализация и работа приемки флагов;
\item инициализация и работа таблицы рейтинга.
\end{itemize}

Результаты работы в данном семестре:

\begin{itemize}
\item доработан существующий API и администраторская панель для управления соревнованиями; 
\item доработан модуль работы с чекерами;
\item доработан модуль приемки флагов;
\item доработан модуль таблицы результатов;
\item созданы тестовые сервисы для тестирования платформы;
\item внедрение системы мониторинга;
\item произведено тестирование платформы в форме соревнований.
\end{itemize}

Пояснительная записка выполнена при помощи системы компьютерной вёрстки \LaTeX.
