Flask - это микрофреймворк, написанный на языке программирования Python, основанный на Werkzeug и Jinja 2. Выпускается под BSD лицензией.

Слово «микро» означает, что цель написания фреймворка - сохранить ядро простым, но в то же время легко расширяемой. По умолчанию Flask не включает в себя уровень абстракции базы данных, валидацию форм. form validation or anything else where different libraries already exist that can handle that. Instead, Flask supports extensions to add such functionality to your application as if it was implemented in Flask itself. Numerous extensions provide database integration, form validation, upload handling, various open authentication technologies, and more. Flask may be “micro”, but it’s ready for production use on a variety of needs.
Особенности фреймворка 
\begin{itemize}
\item удобная работа с URL;
\item богатые возможности шаблонизатора;
\item минимальные требования к ресурсам в сравнении с аналогичными фреймворками;
\end{enumerate}

Flask используется в модуле таблицы рейтинга. 