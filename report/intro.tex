CTF (Capture the flag, Захват флага) --- это командные соревнования, целью которых является оценка умения участников атаковать и защищать компьютерные системы. По типу, соревнования делятся на два типа: task-based (квесты), attack-defense (классические соревнования).

Для проведения соревнований типа attack-defense, каждой команде выдается сервер, на котором имеется ряд сервисов, одинаковые у всех. Сервисами являются программы, которые должны быть запущены на протяжении всего времени соревнований. Раз в минуту жюрейская система проверяет работу сервисов, присылает новый флаг, а также проверяет его наличие на сервере. В роли флага выступает случайно сгенерированная строка, определенной длины. Сервисы заведомо имеют уязвимости, через которые можно украсть флаг. Команда должна найти уязвимости в сервисах и закрыть их. Но также она должна эксплуатировать эти уязвимости на сервисах других команд, похищая флаги. 

Команда Keva имеет большой опыт в проведении соревнований CTF. Во время проведения межрегиональных межвузовских соревнований в области информационной безопасности SibirCTF 2014, 2015 использовались наработки Екатеринбургской команды HackerDom. Однако их решение не подходит нам по некоторым критически важным параметрам. Поэтому  было принято решение написать собственную платформу для проведения соревнований CTF.

Платформа должна отвечать следующим требованиям:
\begin{itemize}
\item Низкое потребление ресурсов сервера;
\item Управление игрой через графический интерфейс;
\item Возможность горизонтального масштабирования.
\end{itemize}

При разработке платформы должны быть использованы уже имеющиеся наработки, в частности, панель администратора и API для task-based проведения соревнований. 

Новая платформа позволит проводить собственные соревнования не используя готовые решения. 
